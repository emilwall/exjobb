% !TEX TS-program = pdflatex
% !TEX encoding = UTF-8 Unicode

\documentclass[12pt]{article}

\usepackage[swedish]{babel} % Svensk typsättning, placera högst upp för kompabilitet med andra paket
\usepackage[utf8]{inputenc} % set input encoding (not needed with XeLaTeX
\usepackage{enumitem} % resume numbering in enumerations
\usepackage{geometry} % to change the page dimensions
\geometry{a4paper} % paper format
\usepackage{graphicx} % support the \includegraphics command and options
\usepackage[parfill]{parskip} % Begin paragraphs with an empty line rather than an indent
\usepackage{verbatim} % adds environment for commenting out blocks of text & for better verbatim
\usepackage{fancyhdr} % This should be set AFTER setting up the page geometry
\pagestyle{fancy} % options: empty , plain , fancy
\usepackage{sectsty} % Section title
\allsectionsfont{\sffamily\mdseries\upshape} % Section font
\usepackage{hyperref} % href

\title{Specifikation av examensarbete}
\author{Emil Wall}
%\date{} % Avkommentera för att dölja dagens datum, eller skriv in datum som ska visas

\begin{document}
\maketitle

\vspace{10mm}

\section{Titel}

Preliminär titel, max 12 ord

\section{Bakgrund}

Här beskrivs i vilket sammanhang examensarbetet skall utföras. Vilka förutsättningar gäller, vad är målet med examensarbetet från uppdragsgivarens synpunkt, vad finns och har gjorts tidigare, och under vilka förhållanden jobbet skall utföras.

\section{Uppgiftsbeskrivning}

Här beskrivs mer detaljerat innehållet i examensarbetet: vad som skall göras och vilka moment ingår. Speciellt skall det beskriva vad som är den intressanta delen i problemet, samt hur detta skall analyseras och lösas. Här bör det alltså framgå att arbetet uppfyller kraven som universitetet har på ett examensarbete på denna nivå.

\section{Tillvägagångssätt}

Vilka system, verktyg och metoder som skall användas. Relevant litteratur (det är oftast en del av arbetet att ta fram ytterligare litteratur). Hur resultatet skall utvärderas och dokumenteras. Studenten ska bidra med information om vilka kurser han/hon har tagit som är relevanta och viktiga för examensarbetet.

\section{Avgränsningar}

Det är också viktigt att skriva ner vad som inte ingår i uppdraget. Detta förebygger att examensarbetet sväller ut okontrollerat. Med fördel kan du skriva ner några punkter som ingår i mån av tid, men räkna med att tiden ofta inte räcker.

\section{Tidsplan}

Här framgår det när jobbet ska göras och hur mycket tid som allokerats till varje moment (noggrannheten kan variera, men inget moment ska vara större än 4 veckor). Tidplanen skall ta hänsyn till faktorer som gör att man inte jobbar heltid med examensarbetet (semester, kurser som ska avslutas, andra åtaganden). Ofta genomförs moment parallellt, t.ex. rapportskrivningen pågår under hela arbetet. Rita gärna en bild som förtydligar sammanhanget mellan olika moment. Rapporteringsmöten bokas in med ämnesgranskaren.

\end{document}



