\pagenumbering{gobble} % Turn off page numbering

\maketitle

\vspace{100pt}
The final version will have title page and endpaper generated from \\
\url{http://pdf.teknik.uu.se/pdf/exjobbsframsida.php} and \\
\url{http://pdf.teknik.uu.se/pdf/abstract.php}. \\
Hence, this page and the abstract are temporary, to be replaced in the final version.

\newpage
\clearpage\mbox{}\clearpage
\newpage

\begin{abstract}
Abstract goes here... Lorem ipsum dolor sit amet, consectetur adipiscing elit. Nam sollicitudin varius libero ac consectetur. Nullam ornare, massa et sagittis consectetur, neque mi scelerisque arcu, in fringilla lectus risus non arcu. Suspendisse vestibulum tellus id mauris lacinia non hendrerit nibh tempor. Proin tempor interdum justo et elementum. Ut ultricies adipiscing ipsum et pharetra. Vestibulum pretium luctus est, quis egestas augue luctus et. Praesent volutpat pharetra lectus vitae elementum.

Integer fringilla ligula eu sem semper tincidunt. Nullam mi lacus, blandit non sollicitudin eget, tempor eu ante. Cum sociis natoque penatibus et magnis dis parturient montes, nascetur ridiculus mus. Morbi ornare sem et purus consequat ac adipiscing nunc tincidunt. Curabitur nisi ante, ornare vel adipiscing et, scelerisque vitae erat. Etiam blandit egestas magna, quis dapibus nulla euismod quis. Sed interdum interdum malesuada. Suspendisse lacinia imperdiet laoreet. Maecenas ullamcorper laoreet nunc ac egestas. Cras consequat elit eu lacus sollicitudin ut pharetra magna venenatis. Suspendisse scelerisque condimentum pulvinar. Mauris ut tellus sit amet nulla porttitor tristique. Suspendisse eleifend erat sed nisi lacinia eu lacinia metus porta. Nulla pretium, risus eget semper laoreet, dolor odio malesuada eros, at mattis enim turpis gravida felis. Aliquam adipiscing varius nibh, ac auctor eros bibendum non.
\end{abstract}

\newpage
\clearpage\mbox{}\clearpage
\newpage

\section*{Popul�rvetenskaplig sammanfattning}

Det kan vara b�de opraktiskt, irriterande och ibland katastrofalt n�r en webbsida eller annan IT-l�sning inte fungerar som den ska. D�rf�r ligger det i tj�nsteleverant�rers intresse att se till att s�dant inte f�rknippas med deras verksamhet, samtidigt som det finns en historia av bristande kvalitetsrutiner inom vissa omr�den av mjukvaruutveckling. Denna uppsats unders�ker vilka sv�righeter som hindrar utvecklare fr�n att skriva automatiska tester f�r sin kod i JavaScript och Standard ML, tv� programmeringsspr�k med intressanta likheter och skillnader. N�gra av de st�rre sv�righeter som identifierats �r att testbarhetsproblem kan uppst� p� grund av hur koden �r skriven, vilken milj� den k�rs i eller att utvecklarna saknar vissa kulturella f�ruts�ttningar f�r att skriva tester. Det �r viktigt att betona att testning i sig inte �r en l�sning p� alla problem och att olika former och niv�er av testning l�mpar sig f�r olika situationer, men att det ocks� kan gynna s�v�l anv�ndare och organisationer i helhet som utvecklarna sj�lva. I takt med att ny serverteknologi, applikationsramverk och nya typer av onlinekurser har introducerats s� har det gjorts framsteg inom testning f�r n�mnda programmeringsspr�k. Denna uppsats ger en �verblick �ver dessa framsteg, och de tekniker som idag anv�nds f�r framg�ngsrik testdriven utveckling.

\newpage
\clearpage\mbox{}\clearpage
\newpage

\section*{Acknowledgment}

Thanks goes to my supervisors Tobias Hasslebrant and Jimmy Larsson for providing me with valuable feedback and connections, to my reviewer Roland Bol for guiding me through the process and giving useful and constructive comments on my work, to the people I have had contact with and interviewed as part of this work, to all my wonderful colleagues at Valtech that never fail to surprise me with their helpfulness and expertise, and to my family and friends (and cats!) for all the little things that ultimately matters the most.

\newpage
\clearpage\mbox{}\clearpage
\newpage

\tableofcontents

\newpage
\clearpage\mbox{}\clearpage
\newpage

\pagenumbering{arabic} % Turn page numbering back on