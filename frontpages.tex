\pagenumbering{gobble} % Turn off page numbering

\maketitle

\vspace{100pt}
The final version will have title page and endpaper generated from \\
\url{http://pdf.teknik.uu.se/pdf/exjobbsframsida.php} and \\
\url{http://pdf.teknik.uu.se/pdf/abstract.php}. \\
Hence, this page and the abstract are temporary, to be replaced in the final version.

\newpage
\null
\newpage

\begin{abstract}
The ever increasing complexity of web applications has brought new demands on automated testing of client side JavaScript. Test-driven development of such applications depends on careful design considerations, selection of tools and frameworks and a favourable testing culture. A contrasting area is that of testing Standard ML, which is also a functional language but with some important differences, notably the static type system and immutability. % embracing a favourable testing culture

The goal with this thesis was to investigate the main problems with testing behaviour of applications written in these two programming languages, and how these problems relate to development tools and practises. What are the testability issues? Which considerations are needed in order to write stable and maintainable tests? How does testing culture affect productivity and quality of software?

Through quantitative interviews, implementation of the DescribeSML testing framework and development of tests in different settings, answers to these questions was sought. Orthogonal challenges for successful testing were identified, combined with inevitably ever present tradeoffs between rigour and simplicity. Recent advancements of technology and frameworks has meant reduced technical limitations whilst demands on software engineering excellence has remained constant if not growing. The main issues are dependency management, how to test graphical interfaces, maintaining separation of concepts, and using the right tools. Harnessing the advancements and applying pragmatism, ingenuity and persistence are key to overcoming these challenges.
\end{abstract}

\newpage
\null
\newpage

\section*{Populärvetenskaplig sammanfattning}

IT-lösningar med många buggar skapar dåliga användarupplevelser. Ingen vill förknippas med det, ändå finns på många håll en historia av bristande kvalitetsrutiner inom mjukvaruutveckling. En aspekt av detta problemområde är vilka svårigheter som hindrar utvecklare från att skriva automatiska tester för sin kod. I denna uppsats ges extra fokus åt JavaScript och Standard ML, två programmeringsspråk med intressanta likheter och skillnader. Det har den senaste tiden skett framsteg inom testning för dessa programmeringsspråk, mycket tack vare att information om hur man går tillväga har blivit mer tillgänglig, nya ramverk gör att kod struktureras på ett annat sätt än tidigare och det har uppstått fler och stabilare sätt att köra koden på. Denna uppsats ger en överblick över dessa framsteg, och de många tekniker som idag används för testdriven utveckling.

Några av de större svårigheter som identifierats är att kod blir olika svår att testa beroende på hur den skrivs och i vilken miljö den körs i. Det spelar även stor roll vilka erfarenhetsmässiga och kulturella förutsättningar utvecklarna har för att skriva tester. Testning i sig är inte en lösning på alla problem och olika former och nivåer av testning lämpar sig för olika situationer, men när det används rätt så kan det gynna både användare, organisationer och utvecklarna själva.

\newpage
\null
\newpage

\section*{Acknowledgment}

Thanks goes to my supervisors Tobias Hasslebrant and Jimmy Larsson for providing me with valuable feedback and connections, to my reviewer Roland Bol for guiding me through the process and giving useful and constructive comments on my work, to the people I have had contact with and interviewed as part of this work, to all my wonderful colleagues at Valtech that never fail to surprise me with their helpfulness and expertise, and to my family and friends (and cats!) for all the little things that ultimately matters the most.

\newpage
\null
\newpage

\tableofcontents

\newpage
\null
\newpage

\pagenumbering{arabic} % Turn page numbering back on
