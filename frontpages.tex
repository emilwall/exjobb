\pagenumbering{gobble} % Turn off page numbering

\maketitle

\vspace{100pt}
The final version will have title page and endpaper generated from \\
\url{http://pdf.teknik.uu.se/pdf/exjobbsframsida.php} and \\
\url{http://pdf.teknik.uu.se/pdf/abstract.php}. \\
Hence, this page and the abstract are temporary, to be replaced in the final version.

\newpage
\null
\newpage

\begin{abstract}
The ever increasing complexity of web applications has brought new demands on automated testing of client side JavaScript. Test-driven development of such applications depends on careful design considerations, selection of tools and frameworks and a favourable testing culture. A contrasting area is that of testing Standard ML, which is also a functional language but with some important differences, notably the static type system and immutability. % embracing a favourable testing culture

The goal with this thesis was to investigate the main problems with testing behaviour of applications written in these two programming languages, and how these problems relate to development tools and practises. What are the testability issues? Which considerations are needed in order to write stable and maintainable tests? How does testing culture affect productivity and quality of software?

Through quantitative interviews, implementation of the DescribeSML testing framework and development of tests in different settings, answers to these questions was sought. Orthogonal challenges for successful testing were identified, combined with inevitably ever present tradeoffs between rigour and simplicity. Recent advancements of technology and frameworks has meant reduced technical limitations whilst demands on software engineering excellence has remained constant if not growing. The main issues are dependency management, how to test graphical interfaces, maintaining separation of concepts, and using the right tools. Harnessing the advancements and applying pragmatism, ingenuity and persistence are key to overcoming these challenges.
\end{abstract}

\newpage
\null
\newpage

\section*{Populärvetenskaplig sammanfattning}

När en webbsida skapas så innebär det ofta flera månader av kodande innan den kan börja användas, i vissa fall flera år. Ju mer avancerad funktionalitet sidan har desto mer tid tenderar utvecklingen att ta i anspråk och desto större blir risken för buggar, särskilt om det är många personer som arbetar med sidan. När sidan väl är lanserad så återstår vanligen en ännu längre tid av drift och underhåll. I det skedet upptäcks och åtgärdas de eventuella buggar som inte hann upptäckas under utvecklingsfasen och det tenderar att ske omfattande förändringar. Det kan vara en stor utmaning att göra detta utan att introducera nya buggar eller förstöra andra delar av sidan, och förändringarna tar ofta längre tid att genomföra än man tror. Att vara noggrann med kvalitetsrutiner och hur en webbsida skrivs kan alltså ses som en investering för att minska kostnader längre fram.

Idag utgör en webbsida ofta första intrycket av ett företag, och sidans skick kan vara avgörande för om det uppstår ett förtroende gentemot företaget. Vissa företag sköter även en stor del av sin centrala verksamhet genom webbsidor så om en sida inte fungerar kan till exempel beställningar eller annan information gå förlorad. Detta innebär att företaget förlorar inkomst, ytterligare en anledning till att undvika buggar.

Ett av de vanligaste sätten att minimera förekomsten av buggar är testning. En variant är att manuellt kontrollera funktioner, till exempel genom att lägga upp varje ny version av en sida på en testserver där man klickar sig runt på webbsidan och kontrollerar att saker fungerar som de ska innan man laddar upp den så att allmänheten kommer åt den. Detta är en billig metod på kort sikt, men det innebär ständig upprepning av tidskrävande procedurer så det kan visa sig ohållbart på längre sikt. Det medför även behov av manuell felsökning för att ta reda på varför något inte fungerar och det kan vara svårt att veta hur något är tänkt att fungera om kraven inte är tillräckligt formaliserade.

Ett alternativ är automatiserade tester, som kan testa specifika delar av koden eller simulera en användares beteende utan mänsklig inblandning. Det är den typen av tester som den här uppsatsen fokuserar på. Mycket av den teknik som används för att skapa hemsidor är förhållandevis ny och det tillkommer hela tiden nya sätt att arbeta på, så det finns ett stort behov av översikt och utvärdering av teknologin. Det finns även mycket att lära genom att jämföra denna teknik med testning i andra sammanhang och i programmeringsspråk som normalt inte används för webben. En fullständig kartläggning av alla tekniker som går att koppla till testning av webbsidor vore dock ett alltför stort projekt för detta format, så denna uppsats är avgränsad till att endast behandla JavaScript och Standard ML, två programmeringsspråk med intressanta likheter och skillnader.

JavaScript används inom webbutveckling för att skapa interaktiva sidor med minimal trafik mot servrar. Standard ML å andra sidan är uppbyggt annorlunda och används oftast i helt andra sammanhang, men stödjer i grunden en liknande programmeringsstil. Det har den senaste tiden skett framsteg inom testning för dessa programmeringsspråk, mycket tack vare att information om hur man går tillväga har blivit mer tillgänglig, nya ramverk gör att kod struktureras på ett annat sätt än tidigare och det har uppstått fler och stabilare sätt att köra koden på. Här ges en överblick över dessa framsteg, och de många tekniker som idag används för testdriven utveckling.

Genom egen utveckling och intervjuer med andra utvecklare har problem och lösningar inom områdets undersökts. Några av de större svårigheter som identifierats är att kod blir olika svår att testa beroende på hur den skrivs och i vilken miljö den körs i. Det spelar även stor roll vilka erfarenhetsmässiga och kulturella förutsättningar utvecklarna har för att skriva tester. Testning i sig är inte en lösning på alla problem och olika former och nivåer av testning lämpar sig för olika situationer, men när det används rätt så kan det gynna både användare, organisationer och utvecklarna själva.

\newpage

\section*{Acknowledgment}

Thanks goes to my supervisors Tobias Hasslebrant and Jimmy Larsson for providing me with valuable feedback and connections, to my reviewer Roland Bol for guiding me through the process and giving useful and constructive comments on my work, to the people I have had contact with and interviewed as part of this work, to all my wonderful colleagues at Valtech that never fail to surprise me with their helpfulness and expertise, and to my family and friends (and cats!) for all the little things that ultimately matters the most.

\newpage
\null
\newpage

\tableofcontents

\newpage
\null
\newpage

\pagenumbering{arabic} % Turn page numbering back on
